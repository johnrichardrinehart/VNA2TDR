\section{Methods of Implementation}
\label{sec:methods of implementation}

This section will outline the various approaches that I'm using to solve this
problem. Where appropriate, I will indicate which Git branch I am using for each
implementation.
\subsection{Na\"{i}ve Implementation - [master]}
\label{sub:naive_implemenation_master}



In this approach, the idea is simple:

\begin{enumerate}
  \item DFT the time-domain signal whose response you're
    interested in, obtaining the analytic frequency-domain function
    \subitem $\mathcal{FFT}(v(t)) = S(\omega)$.
  \item Multiple the VNA data and the sampled version of the signal (frequency
    domain) to obtain the response
    \subitem $ S(\omega)\textrm{VNA}(\omega) = R(\omega) $
  \item Invert the response to obtain the time-domain version:
    \subitem $\mathcal{DFT}^{-1}(R(w)) = r(t)$
\end{enumerate}

Variations on this involve using $S(w) = 1$ (a delta function in time) and
convolving the "impulse response" with the time-domain signal $ s(t) $.

\begin{figure}[h]
  \centering
  \begin{tikzpicture}
    [scale=.8,auto=left,every node/.style={rectangle,fill=purple!20,rounded
    corners=3pt}]
    % upper branch
    \node (a) at (0,0) {Acquire $ \textrm{VNA}(f) $};
    \node (b) [right=of a] {Generate $S(\omega)$};
    \node (c) [right=of b] {$R(\omega) = \textrm{VNA}(\omega)S(\omega)$};
    \node (d) [right=of c] {$\mathcal{DFT}^{-1}(\omega)(R(\omega)) = r(t)$};
    \draw[->] (a) -- (b); \draw[->] (b) -- (c); \draw[->] (c) -- (d);
    % Lower branch
    \node (e) [below right=of a] {$ \mathcal{DFT}^{-1}(\textrm{VNA}(f)) =
      \textrm{vna}(t) $};
    \node (f) [right=of e] {$ \textrm{vna(t)} * s(t) = r(t)$};
    \draw[->] (a) -- (e.west); \draw[->] (e) -- (f);
  \end{tikzpicture}
  \caption{Workflow for approach 1.}
  \label{fig:approach_1_graph}
\end{figure}

\subsubsection{Caveats: Approach 1}
\label{sub:caveats_approach_1}

\begin{enumerate}
  \item VNA data is assumed to be periodic, but $ S(\omega) $ is not periodic
    ($\mathcal{F}$).
  \item Time-domain points (spacing, number) are constrained by the frequency-domain points.
\end{enumerate}

\subsection{Phillip Dunsmore's Approach - [pdunsmore]}
\label{sub:phillip_dunsmore_s_approach_pdunsmore_}

Phillip Dunsmore's thesis provides a mathematical framework for approaching this
problem. His solution most differs from the na\"{i}ve solution in that the VNA
data is assumed to have been acquired as if a sampling function was applied to
the frequency domain. He uses the sampling function $\Sh(f)$ to pluck off
frequency components. He accounts for the finite frequency domain by applying a
rectangular filter function $\Theta(f_{max})$ to the measured data.

Concretely,

\begin{enumerate}
  \item
\end{enumerate}
